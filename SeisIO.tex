\documentclass[12pt]{article} % A4 paper and 11pt font size

%Bibliography
\usepackage{natbib}
\usepackage{fullpage}
\usepackage[hidelinks]{hyperref}
\usepackage{setspace}
\usepackage{multirow}
\usepackage[english]{babel} % English language/hyphenation
\usepackage{amsmath,amsfonts,amsthm} % Math packages
\usepackage[dvipdf]{graphicx}
\usepackage{times,graphicx,pstricks}
\usepackage{epstopdf}
\usepackage{float}
\usepackage[font={footnotesize}]{caption} %Caption smaller
\usepackage{lineno}
\usepackage{authblk}
\usepackage{caption}
\usepackage{lipsum}
\doublespacing
\linenumbers*[1]
\setlength\parindent{0pt}
\setlength{\parskip}{.75em}
\def\bibfont{\footnotesize} % Biblio in small
\setlength{\bibsep}{10pt} %space between citations

%----------------------------------------------------------------------------------------
%	TITLE SECTION
%----------------------------------------------------------------------------------------

\title{SeisIO: a fast, efficient geophysical data architecture for the Julia language\\
	\vspace{1.0cm}}

\author{Joshua Jones$^{1\ast}$, Tim Clements$^{2}$, Kurama Okubo$^{2}$, and Marine A. Denolle$^{2}$\\ 
	\vspace{0.4cm}
	\normalsize{$^{1}$I have no idea what to list here.\\
		$^{2}$Department of Earth and Planetary Sciences, Harvard University, MA, USA}\\
	
	\vspace{2.0cm}
	\normalsize{$^\ast$Corresponding author: Joshua Jones (jpjones.gphys@gmail.com)}}
\date{}

\begin{document}
	\maketitle
	\pagenumbering{arabic}  
	
	\newpage
	
	\section*{Abstract}
	\lipsum[1]
	
	
	\newpage
	\section{Introduction}
	The Julia language, developed for fast, efficient numerical computing, was released August 2018 and has been in development since 2009 \citep{Bezanson2017, Bezanson2018}. It combines the ease of high-level languages like MATLAB and Python with excellent performance: while still in beta testing, it became the fourth programming language to achieve a petaflop, after Fortran, C, and C++ \citep{Reiger2018}.
	
	% 
	% will add more later
	% 
	
	\section{SeisIO}
	The SeisIO package was created in 2016 with the intention of facilitating rapid, efficient analysis of univariate geophysical data in the Julia language. The design allows the user to read disparate types of univariate geophysical data into a single structure, including gapped and irregularly-sampled data.
	
	SeisIO includes well-tested read support for a number of geophysical time-series formats, including Ad-Hoc (AH), Adaptable Seismic Data Format \citep[ASDF; ][]{Krischer2016}, GeoCSV time-sample pair (tspair) and sample list (slist), mini-SEED, SEG Y and the PASSCAL variant, SAC, SUDS \citep{Ward1989}, UNAVCO Bottles, University of Washington (UW), and WIN \citep{NIED2019}. Metadata support includes readers for station and event XML \citep[e.g., ][]{Schorlemmer2012}, SAC pole-zero files, and dataless SEED response files. Standard data processing operations supported by SeisIO include removal of the mean and linear trend, bandpass filtering, instrument response translation and removal (i.e., flattening to DC), resampling, cosine tapering, merging, seismogram differentiation/integration, and time synchronization. Online tools for data acquisition support the three standard FDSN protocols, IRIS timeseries requests, the IRIS TauP interface \citep{Crotwell1999}, and SeedLink.
	
	SeisIO has been part of the Julia package ecosystem since early 2019. Automated testing in travis-ci and appveyor supports installation in Linux, Mac OS, and Windows. Code coverage estimates of 98\% on CodeCov.io and Coveralls exceed the standards of enterprise-level software releases.
	
	% 
	% will add more later
	% 
	
	\section{Benchmarking}
	
	To investigate computational speed and efficiency, we conduct benchmark tests using a 64-bit desktop computer equipped with an Intel DH67CL motherboard, i7-2600 (3.4 GHz) CPU, and 16 GB Kingston DDR3 RAM, running Julia v1.1.0 on 64-bit Ubuntu Linux 18.04.3 (kernel 5.0.0-29). 
	
	File read tests are described in Table \ref{tab:bench_data}.  Tests use SeisIO v0.4.0 and BenchmarkTools.jl, with 100 samples per benchmark and one evaluation per sample. Because the Julia language uses a JIT compiler, an initial "precompile" run precedes each test. Results appear in Table \ref{tab:bench_julia_tabular} and Fig. \ref{fig:bench_raw}.
	
	We define the memory overhead of each benchmark as $\textrm{obj\_sz}/\textrm{file\_sz} - 1.0$, where obj\_sz is object size in memory (headers plus data) and file\_sz is size on disk. In most cases overhead is negligible. File read times of GeoCSV are large because the format stores all numbers as ASCII strings.
	
	\subsection{Comparative Performance}
	We now compare the performance of SeisIO to two popular, well-established data architectures: ObsPy \citep{Beyreuther2010,Megies2011} and SAC \citep{Goldstein2003, Goldstein2005}. In both cases, we perform all tests in Table \ref{tab:bench_data} for which file readers exist. Comparative memory use is shown in Fig. \ref{fig:memory} and read times are shown in Fig. \ref{fig:time}. ObsPy v1.1.1 is tested in a dedicated Python 3.7.3 environment created with conda 4.7.12, using memory-profiler 0.55.0 and timeit.py; comparisons are limited because ObsPy has no data reader for SUDS, UW, or PASSCAL SEG Y, and neither GeoCSV nor Win32 v1 parses correctly. SAC v106.a is tested with perf v5.0.21 for task-clock monitoring and time -v for memory. The time and memory associated with starting and exiting SAC (without executing any commands) was estimated with perf and removed. In all test cases, SeisIO uses less memory and reads files more quickly. 
	
	%	\begin{table}[h]
	%		\caption[]{ObsPy v1.1.1 benchmarks. Columns have the same meanings as in Table \ref{tab:bench_julia_tabular}. The SAC-full test uses KW debug=True.}\label{tab:bench_py_tabular}
	%		\begin{tabular}{lrrrr}
	%			\textbf{Test Name} & \textbf{Sz {[}MB{]}} & \textbf{Mem {[}MB{]}} & \textbf{Ovh {[}\%{]}} & \textbf{T {[}ms{]}} \\
	%			AH            & 0.66                  & 135.82                 & 20338.65               & 65.53                \\
	%			MSEED-1       & 32.96                 & 388.89                 & 1079.82                & 106.8                \\
	%			MSEED-2       & 5.65                  & 246.84                 & 4271.6                 & 36.12                \\
	%			SAC           & 32.96                 & 409.67                 & 1142.84                & 58.72                \\
	%			SAC-full      & 32.96                 & 409.71                 & 1142.96                & 58.19               
	%		\end{tabular}
	%	\end{table}
	%	
	%	\begin{table}[h]
	%		\caption[]{SAC v101.6a benchmarks. Columns have the same meanings as in Table \ref{tab:bench_julia_tabular}.}\label{tab:bench_sac_tabular}
	%		\begin{tabular}{lrr}
	%			\textbf{Test Name} & \textbf{Mem {[}MB{]}} & \textbf{T {[}ms{]}} \\
	%			SAC-full & 66.02 & 113.58 \\
	%			SUDS     & 5.46  & 15.35 
	%		\end{tabular}
	%	\end{table}
	
	\section{Tutorial}
	To add: info. on Tim's tutorial? Can that be revised? Should Josh add that to GitHub? It was excellent.
	
	\section{Conclusions}
	\lipsum[1]
	
	
	\section*{Acknowledgements}
	The authors thank Andy Nowacki (University of Leeds, UK) for discussions on the Julia language; Douglas Neuhauser (University of California Seismological Laboratory, USA) for discussions of the SAC data format; and Chad Trabant (Incorporated Research Institutions for Seismology, USA), Robert Casey (Incorporated Research Institutions for Seismology, USA), Ellen Yu (California Institute of Technology, USA), and Aparna Bhaskaran (California Institute of Technology, USA) for assistance with FDSN and correspondence. J. Jones extends his thanks to Wendy McCausland (USGS-VDAP, USA) and Ken Creager (University of Washington, USA) for contributing test data, and R. Carniel (Universita di Udine, Italy) for early testing. mini-SEED handling was originally based on rdmseed.m for MATLAB by Francois Beauducel (Institut de Physique du Globe de Paris, France); SAC routines were originally based on SacIO.jl by Ben Postlethwaite. This research was supported by a grant from the Packard Foundation. 
	
	\subsection*{Data Availability}
	Data used in benchmark tests can be found in the SeisIO GitHub repository with exceptions given in Table \ref{tab:bench_data}. Benchmarking scripts are available on the SeisIO GitHub page. 
	% 
	% *** they aren't actually on GitHub yet. I'll place these in a private GitHub repository late this week and add you all to that, so that all coauthors have access. I'll move them into SeisIO and make them public after the manuscript is submitted.
	% 
	% \bibliography{quakebib.bib}
	
	
	\begin{thebibliography}{}
		%
		% *** Placeholder, will add BiBTeX files later.
		%
		\itemsep 6pt
		\bibitem[Beyreuther et al, 2010]{Beyreuther2010}M. Beyreuther, R. Barsch, L. Krischer, T. Megies, Y. Behr and J. Wassermann (2010), ObsPy: A Python Toolbox for Seismology, SRL, 81(3), 530-533. DOI: 10.1785/gssrl.81.3.530
		
		\bibitem[Bezanson et al., 2017]{Bezanson2017}Bezanson, J., Edelman, A., Karpinski, S., \& Shah, V. B. (2017). Julia: A fresh approach to numerical computing. SIAM review, 59(1), 65-98.
		
		\bibitem[Bezanson et al., 2018]{Bezanson2018}Bezanson, J., Chen, J., Chung, B., Karpinski, S., Shah, V. B., Vitek, J., \& Zoubritzky, L. (2018). Julia: dynamism and performance reconciled by design. Proceedings of the ACM on Programming Languages, 2(OOPSLA), 120.
		
		\bibitem[Crofwell et al., 1999]{Crotwell1999}Crotwell, H. P., T. J. Owens, and J. Ritsema (1999). The TauP Toolkit: Flexible seismic travel-time and ray-path utilities, Seismological Research Letters 70, 154?160.
		
		\bibitem[Goldstein and Snoke, 2005]{Goldstein2005}Goldstein, P., A. Snoke, (2005), ?SAC Availability for the IRIS Community?, Incorporated Institutions for Seismology Data Management Center Electronic Newsletter.
		
		\bibitem[Goldstein et al., 2003]{Goldstein2003}Goldstein, P., D. Dodge, M. Firpo, Lee Minner (2003) ?SAC2000: Signal processing and analysis tools for seismologists and engineers, Invited contribution to ?The IASPEI International Handbook of Earthquake and Engineering Seismology?, Edited by WHK Lee, H. Kanamori, P.C. Jennings, and C. Kisslinger, Academic Press, London.
		
		\bibitem[Jones and Malone, 2005]{Jones2005}Jones, J.P., \& Malone, S. D. (2005). Mount Hood earthquake activity: Volcanic or tectonic origins?. Bulletin of the Seismological Society of America, 95(3), 818-832.
		
		\bibitem[Krischer et al., 2016]{Krischer2016}Lion Krischer, James Smith, Wenjie Lei, Matthieu Lefebvre, Youyi Ruan, Elliott Sales de Andrade, Norbert Podhorszki, Ebru Bozda{\"g}, Jeroen Tromp, An Adaptable Seismic Data Format, Geophysical Journal International, Volume 207, Issue 2, November, 2016, Pages 1003?1011, https://doi.org/10.1093/gji/ggw319.
		
		\bibitem[Megies et al., 2011]{Megies2011}T. Megies, M. Beyreuther, R. Barsch, L. Krischer, J. Wassermann (2011), ObsPy ? What can it do for data centers and observatories?, Annals Of Geophysics, 54(1), 47-58, DOI: 10.4401/ag-4838.
		
		\bibitem[NIED, 2019]{NIED2019}National Research Institute for Earth Science and Disaster Resilience (2019), NIED Hi-net, National Research Institute for Earth Science and Disaster Resilience, doi:10.17598/NIED.0003.
		
		\bibitem[Reiger et al., 2018]{Reiger2018}Regier, J., Fischer, K., Pamnany, K., Noack, A., Revels, J., Lam, M., Howard, S., Giordano, R., Schlegel, D., McAuliffe, J. and Thomas, R., 2019. Cataloging the visible universe through Bayesian inference in Julia at petascale. Journal of Parallel and Distributed Computing, 127, pp.89-104.
		
		\bibitem[Schorlemmer et al., 2012]{Schorlemmer2012}Schorlemmer, D., Euchner, F., K�stli, P., \& Saul, J. (2011). QuakeML: status of the XML-based seismological data exchange format. Annals of Geophysics, 54(1), 59-65.
		
		\bibitem[Ward, 1989]{Ward1989}Ward, Peter L. (1989). SUDS; seismic unified data system, USGS Open-File Report 89-188, doi:10.3133/ofr89188.
		
	\end{thebibliography}
	\newpage
	\begin{table}[H]
		\small
		\captionsetup{singlelinecheck=off}
		\caption[]{Benchmark tests. Columns: Test Name is how the test is referenced in this manuscript; Filename is the name or search pattern in SeisIO/test/SampleFiles/; Sz is file size in MB;  Src is data source, given below (for request parameters, please contact the corresponding author).
			\begin{enumerate}
				\item contributed by Prof. K. Creager, University of Washington, USA.
				\item contributed by M. Denolle.
				\item retrieved with IRIS FDSN dataselect.
				\item from IRIS \_STHELENS-1980 virtual network and data set; available from IRIS, USA.
				\item redistribution restricted; to request this file please contact Dr. W. McCausland, USGS-VDAP.
				\item extracted from Pacific Northwest Seismic Network archives; data from \cite{Jones2005}. 
				\item data from HiNet \citep{NIED2019}; redistribution prohibited.
			\end{enumerate}
		}\label{tab:bench_data}
		\begin{tabular}{lllll}
			\textbf{Test Name} & \textbf{Filename}           & \textbf{Format} & \textbf{Sz {[}MB{]}} & \textbf{Src} \\
			AH                 & 20050904.PA01.E.sac.ah      & AH              & 0.14                 & 1\\
			ASDF               & 2019\_07\_07\_00\_00\_00.h5 & ASDF      	   & 21.96                & 2\\
			GeoCSV-tspair      & FDSNWS.IRIS.geocsv          & GeoCSV tspair   & 3.31                 & 3\\
			GeoCSV-slist       & geocsv\_slist.csv           & GeoCSV slist    & 8.25                 & 3\\
			MSEED-1            & one\_day.mseed              & mini-SEED       & 19.09                & 2\\
			MSEED-2            & SHW.UW.mseed                & mini-SEED       & 1.79                 & 4\\
			PASSCAL            & test\_PASSCAL.segy          & PASSCAL SEG Y   & 32.96                & 3\\
			SAC                & one\_day.sac                & SAC             & 32.96                & 2\\
			SUDS               & 10081701.WVP                & SUDS            & 1.26                 & 5\\
			UW                 & 99011116541W                & UW              & 23.15                & 6\\
			WIN                & 2014092709*.cnt             & WIN             & 4.49                 & 7\\
		\end{tabular}
	\end{table}
	
	\begin{table}[H]
		\small
		\caption[]{SeisIO v0.4.0 benchmarks. Columns: Sz is size of object in memory; Mem is total memory use; Ovh is memory overhead (defined in text); T is read time in ms; for Notes, 1 = test uses SeisIO read\_hdf5; 2 = test uses SeisIO read\_data with KW full=true; 3 = test uses SeisIO read\_data with KW nx\_new=36000, nx\_add=1400000. Other tests use SeisIO read\_data with default parameters.}\label{tab:bench_julia_tabular}
		\begin{tabular}{lrrrrl}
			\textbf{Test Name} & \textbf{Sz {[}MB{]}} & \textbf{Mem {[}MB{]}} & \textbf{Ovh {[}\%{]}} & \textbf{T {[}ms{]}} & \textbf{Notes} \\
			AH                 & 0.33                 & 0.33                  & 1.16                  & 0.63                &                \\
			ASDF               & 26.37                & 26.49                 & 0.45                  & 93.92               & 1              \\
			GeoCSV-tspair      & 0.39                 & 0.44                  & 12.30                 & 209.84              &                \\
			GeoCSV-slist       & 6.80                 & 6.87                  & 1.07                  & 123.08              &                \\
			MSEED-1            & 32.96                & 32.96                 & 0.01                  & 71.06               &                \\
			MSEED-2            & 5.35                 & 6.19                  & 15.75                 & 9.53                & 3              \\
			PASSCAL            & 32.96                & 32.99                 & 0.08                  & 22.42               & 2              \\
			SAC                & 32.96                & 32.97                 & 0.04                  & 12.88               & 2              \\
			SUDS               & 2.53                 & 2.59                  & 2.43                  & 1.36                &                \\
			UW                 & 37.66                & 40.29                 & 6.98                  & 26.9                &                \\
			WIN                & 10.99                & 11.25                 & 2.33                  & 24.67               &               
		\end{tabular}
	\end{table}
	
	\begin{figure}[H]
		\centering
		\includegraphics[width=1.0\linewidth]{raw_benchmarks.png}
		\caption{Benchmarks tests (Table \ref{tab:bench_data}) in Julia v1.1.0 with SeisIO v0.4.0. Left: file read times. Right: peak memory use in SeisIO and file size on disk.} \label{fig:bench_raw}
	\end{figure}
	
	\begin{figure}[H]
		\centering
		\includegraphics[width=1.0\linewidth]{mem_hist.png}
		\caption{Memory use and overhead for all benchmarks in Table \ref{tab:bench_data} that were testable in at least two of ObsPy, SAC, and SeisIO. (top) Memory usage and file sizes on disk. (bottom) Memory overhead. The \textit{y}-axis is logarithmic. A missing bar with text label NR" indicates no reader.} \label{fig:memory}
	\end{figure}
	
	\begin{figure}[H]
		\centering
		\includegraphics[width=1.0\linewidth]{time_hist.png}
		\caption{Read times in milliseconds for all benchmarks in Table \ref{tab:bench_data} that were testable in at least two of ObsPy, SAC, and SeisIO. A missing bar with text label NR" indicates "no reader".} \label{fig:time}
	\end{figure}
	
\end{document}